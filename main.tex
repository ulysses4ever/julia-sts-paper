\documentclass[sigplan,review]{acmart}

\settopmatter{printfolios=true,printccs=false,printacmref=false}

\usepackage[utf8]{inputenc}

% ********************************************************************
%
%        ACM / SIGPLAN Stuff
%
% ********************************************************************

%% Copyright information
%% Supplied to authors (based on authors' rights management selection;
%% see authors.acm.org) by publisher for camera-ready submission;
%% use 'none' for review submission.
%\setcopyright{acmcopyright}
%\setcopyright{acmlicensed}
%\setcopyright{rightsretained}
%\copyrightyear{2017}           %% If different from \acmYear
%%% If you see 'ACMUNKNOWN' in the 'setcopyright' statement below,
%%% please first submit your publishing-rights agreement with ACM (follow link on submission page).
%%% Then please reload our instructions page and copy-and-paste the NEW commands into your article.
%%% Please contact us in case of questions; allow up to 10 min for the system to propagate the information.
%%%
%%% The following is specific to VMIL '23 and the paper
%%% 'Approximating Type Stability in the Julia JIT'
%%% by Artem Pelenitsyn.
%%%
\copyrightyear{2023}
\acmYear{2023}
\setcopyright{acmlicensed}\acmConference[VMIL '23 ]{Proceedings of the 15th
ACM SIGPLAN International Workshop on Virtual Machines and Intermediate
Languages }{October 23, 2023}{Cascais, Portugal}
\acmBooktitle{Proceedings of the 15th ACM SIGPLAN International Workshop on
Virtual Machines and Intermediate Languages (VMIL '23 ), October 23, 2023,
Cascais, Portugal}
\acmPrice{15.00}
\acmDOI{10.1145/3623507.3623556}
\acmISBN{979-8-4007-0401-7/23/10}

\acmSubmissionID{splashws23vmilmain-p99-p}
\received{2023-07-23}
\received[accepted]{2023-08-28}


% ********************************************************************
%
%        Various Convenience Packages
%
% ********************************************************************
\usepackage[utf8]{inputenc}
\usepackage[T1]{fontenc}
\usepackage{tcolorbox}
\usepackage{amsmath,amsthm}

\usepackage{xspace,url,hyperref,doi,wrapfig,stmaryrd,graphicx,xparse,etoolbox}
\usepackage{caption,subcaption}
\usepackage[inline]{enumitem}
\usepackage{cancel}
\usepackage[xcolor,rightbars]{changebar}
%\usepackage{showframe}  %% for checking margins
\definecolor{Gray}{gray}{0.9}
\definecolor{vlightgray}{gray}{0.93}
\usepackage{bigfoot} % verbatim in footnote
\interfootnotelinepenalty=10000 % don't break footnotes into several pages

% ********************************************************************
%
%        Bibliography
%
% ********************************************************************
\bibliographystyle{ACM-Reference-Format}


% ********************************************************************
%
% Fine-tune hyperreferences (hyperref should be called last)
%
% ********************************************************************
% \hypersetup{%
%   %draft, % hyperref's draft mode, for printing see below
%   colorlinks=true, linktocpage=true, pdfstartpage=3, pdfstartview=FitV,%
%   % uncomment the following line if you want to have black links (e.g., for printing)
%   %colorlinks=false, linktocpage=false, pdfstartpage=3, pdfstartview=FitV, pdfborder={0 0 0},%
%   breaklinks=true, pageanchor=true,%
%   pdfpagemode=UseNone, %
%   % pdfpagemode=UseOutlines,%
%   plainpages=false, bookmarksnumbered, bookmarksopen=true, bookmarksopenlevel=1,%
%   hypertexnames=true, pdfhighlight=/O,%nesting=true,%frenchlinks,%
%   urlcolor=CTurl, linkcolor=CTlink, citecolor=CTcitation, %pagecolor=RoyalBlue,%
%   %urlcolor=Black, linkcolor=Black, citecolor=Black, %pagecolor=Black,%
%   pdftitle={\myTitle},%
%   pdfauthor={\myName},%
%   pdfsubject={\myTitle},%
%   pdfkeywords={\myKws},%
%   pdfcreator={pdfLaTeX},%
%   pdfproducer={LaTeX with hyperref and classicthesis}%
% }

% *********************************************************************************************
%
%               Setup code listings
%
% *********************************************************************************************

\usepackage{style/julia}
\usepackage{mathpartir,listings}
\lstdefinelanguage{Jules}{
  keywords={struct,is,end},
  keywordstyle=\color{darkgray}\bfseries,
  ndkeywords={struct,is,end},
  ndkeywordstyle=\color{darkgray}\bfseries,
  identifierstyle=\color{black},
  sensitive=false,  comment=[l]{//},  morecomment=[s]{/*}{*/},
  commentstyle=\color{gray}\ttfamily,  stringstyle=\color{gray}\ttfamily,
  morestring=[b]',  morestring=[b]",
  aboveskip=\medskipamount, %0em,
  belowskip=\medskipamount, %0em
  escapeinside={(*@}{@*)}
}
\lstset{
  language=Jules,  extendedchars=true,  basicstyle=\small\ttfamily,
  showstringspaces=false,   showspaces=false,  numberstyle=\small,
  numbersep=9pt,  tabsize=2, breaklines=true,  showtabs=false, captionpos=b
}

\newcommand{\code}[1]{{\ttfamily #1}\xspace}
\renewcommand{\c}[1]{\lstinline[language=Julia]!#1!\xspace}

\input{macros}

\begin{document}
\title{Approximating Type Stability in the Julia JIT}
\begin{abstract}
Julia is a dynamic language for scientific computing. For a dynamic language,
Julia is surprisingly typeful. Types are used not only to structure data but
also to guide dynamic dispatch – the main design tool in the language. No
matter the dynamism, Julia is performant: flexibility is compiled away at the
run time using a simple but smart type-specialization based optimization
technique called type stability. Based on a model of a JIT mimicking Julia
from previous works, we present the first algorithm to approximate type
stability of Julia code. Implementation and evaluation of the algorithm is
still a work in progress.
\end{abstract}

\begin{CCSXML}
<ccs2012>
<concept>
<concept_id>10011007.10011006.10011041.10011044</concept_id>
<concept_desc>Software and its engineering~Just-in-time compilers</concept_desc>
<concept_significance>500</concept_significance>
</concept>
</ccs2012>
\end{CCSXML}

\ccsdesc[500]{Software and its engineering~Just-in-time compilers}

\keywords{method dispatch, type inference, compilation,%
  dynamic languages, Julia language}


\maketitle

\section{Introduction}

The scientific computing community has long been of two minds about the language
technology best suited for the task. On the one hand, the exploratory nature of
programming with raw datasets and mathematical models with many parameters calls
for modern convenience features like dynamic typing and garbage collection. On
the other, crunching numbers may require to squeeze every CPU cycle available.
Accordingly, the language market in this area is split into productivity
languages (Python, MATLAB, R) and high-performance oriented languages (C,
C++, Fortran).

The splitted nature of the language landscape in the area of scientific
computing calls for a question: can we have the best of both worlds? Julia
attempts to answer in the positive.

On the surface, the Juila language has a lot of pythonistic feeling to it.
Consider the following function that sums an array of elements but clamps the
summands above the given threshold.
%
\begin{verbatim}
  function sum(v, t)
    res = v[1]
    for i = 2:length(v)
      elm  = v[i] < t ? v[i] : t
      res = res + elm
    end
    res
  end
\end{verbatim}
%
What strikes about this code is the absence of type annotations either on the
function arguments or anywhere locally. Of course, it is possible to call
\c{sum} with a string and a number and get a runtime failure. On the other hand,
and no less of a surprise, when called with the right arguments the code gets
JIT-compiled and performs on par with equivalent C code.

Julia's competitive performance is documented both in the language manual and in
academic papers~\cite{oopsla18a}. One of the key techniques enabling
essential optimizations is based on the code property called \emph{type
stability}~\cite{Pelenitsyn21}. While well-understood on the IR level, the
property has been elusive for end users: it is mentioned in the manual and on
many forum threads but still does not have a clear source-level model for it.
In this short paper we propose the first algorithm to approximate this highly
dynamic property statically. In particular, we
\begin{itemize}

  \item give an informal description of type stability as it comes up in
  practice of Julia programming (\secref{sec:back}; largely following~\cite{Pelenitsyn21});

  \item discuss challenges to model type stability statically and the relation
  of the task to the full-fledged type inference;

  \item give an algorithm to approximate type stability statically;

  \item discuss termination conditions of the algorithm, and an approach for
  evaluating it.
\end{itemize}

% \section{A Typeful Language in The Untyped Universe}%
\section{A Type Stability Primer}%
% \section{Julia: Productivity And Performance Reconciled}
\label{sec:back}

\subsection{Multiple Dispatch in Julia}%

The Julia language is designed around multiple dispatch~\cite{BezansonEKS17}.
Programs consist of \emph{functions} that are implemented by multiple
\emph{methods}; there is nothing more to a Julia function than just a name. Each
method is identified by a distinct type signature. At run time, the Julia
implementation dispatches a function call to the \emph{most specific} method by
comparing the types of the arguments to the types of the parameters of all
methods of that function. For example, the call to the \c{+} function in the
example from the introduction can dispatch to one out of over two hundreds
methods for \c{+} in the standard library alone (packages can add more methods
to a function). % TODO: \figref{fig:plus}.

% TODO: figure how to make figure* work...
% \begin{figure*}
% \begin{lstlisting}[language=julia]
% # 184 methods for generic function "+":
% [1] +(x::T, y::T) where T<:Union{Int128, Int16, Int32, Int64, Int8, UInt128, UInt16, UInt32, UInt64, UInt8} in Base at int.jl:87
% [2] +(x::T, y::T) where T<:Union{Float16, Float32, Float64} in Base at float.jl:383[3] +(::Missing) in Base at missing.jl:100
% ...
% \end{lstlisting}
% \caption{Methods from the standard library}\label{plus}\label{fig:plus}
% \Description{Several of 206 methods for the \c{+} function in the Julia standard
%   library as of Julia 1.8.5}
% \end{figure*}


Julia supports a rich type language for defining method signatures. Base types
consist of either bits types (i.e. types with a direct binary representation,
like integers) or record types (called structs in Julia). Both bits types and
record types, referred to as \emph{concrete types}, can have supertypes, but all
supertypes are \emph{abstract types}. Abstract types are arranged into a
subtyping hierarchy rooted at the built-in \c{Any} type. Every value in the
program have a unique type tag that can be accessed via the \c{typeof} funciton.
%
The type language allows for further composition of these base
types using unions, tuples, and bounded existential constructors; the result of
composition can be abstract or concrete. \citet{oopsla18b} gives a detailed
discussion of the type language and of subtyping.

Any function call in a program, such as \c{res+elm} in the example above,
requires choosing one of the methods of the target function. \emph{Method
dispatch} is a multi-step process. First, the implementation obtains the
concrete types of arguments. Second, it retrieves applicable methods by checking
for subtyping between argument types and type annotations of the methods. Next,
it sorts these methods into subtype order. Finally, the call is dispatched to
the most specific method---a method such that no other applicable method is its
strict subtype. If no such method exists, an error is produced.

Function calls are pervasive in Julia, and their efficiency is crucial for
performance. However, the multi-step dispatch mechanism
make the process slow. To attain acceptable performance, the compiler attempts
to remove as many dispatch operations as it can. This optimization leverages
run-time type information whenever a method is compiled, i.e., when it is called
for the first time with a novel set of argument types.  These types are used by
the compiler to infer types in the method body. Then, this type information
frequently allows the compiler to devirtualize and inline the function calls
within a method~\cite{aigner}, thus improving performance. However, this
optimization is not always possible: if type inference cannot produce a
sufficiently specific type, then the call cannot be devirtualized.

\subsection{Type Stability: A Key To Performance?}

To illustrate a profound effect that type inference precision can have on
performance, consider the \c{sum} function from the introduction benchmarked in
three scenarios differing only in input types. Assuming a random array of integers
called \c{vint} and a random array of floating-point numbers called \c{vflt}
(both consisting of 10K elements)
compare the median running time of the following three calls to \c{sum}:
\begin{itemize}

  \item \c{sum(vint, 0)}~--- 1.397 microseconds
  \item \c{sum(vflt, 0.5)}~--- 11.489 microseconds
  \item \c{sum(vint, 0.5)}~--- 64.623 microseconds
\end{itemize}

When a performance regression occurs, it is common for Julia developers to study
the intermediate representation produced by the compiler. To facilitate this,
the language provides a macro, \c{@code_warntype}, that shows Julia's
intermediate representation of the code along with the inferred types for a given function
invocation. Types that are imprecise, i.e. abstract, show up in red: they
indicate that concrete type of a run-time value may vary from run to run.
%
In the first two benchmarks, the Julia compiler is able to deduce a concrete return
type of the method (\c{Int} and \c{Float} correspondingly), but the type of the last one
reported as \c{Union\{Int, Float\}}, which is an abstract type.
%
Such type imprecision can impact performance in two ways. First,
the \c{res} variable has to be boxed, adding a level of indirection to
any operation performed therein. Second, it is harder for
the compiler to devirtualize and inline consecutive calls, thus requiring
dynamic dispatch.

Julia's compilation model is designed to accommodate source programs
with flexible types. Yet, to make such programs efficient, the compiler
creates an \emph{instance} of each source method for each distinct tuple of
argument types. Thus, even if the programmer does not provide any type
annotations, like in the \c{sum} example, the compiler will create method
instances for \emph{concrete} input types seen during
an execution. For example, the three benchmarks shown above will make the
compiler create three distinct method instances.
Because method instances have more precise argument types, the compiler can
leverage them to produce more efficient code and infer more precise return types.

In Julia parlance, a method is called \emph{type stable} if, given a concrete
input type, it is possible to infer a concrete output type. The \c{sum} function
is not type stable because for the input type \c{Tuple\{Vector\{Int\},
Float64\}}\footnote{Tuple types encode several input arguments in Julia.}
the return type can be either \c{Int} or \c{Float64}.

\bibliography{bib/jv,bib/all,bib/lj}

\end{document}
