\author{Artem Pelenitsyn}
\affiliation{
  %~ \position{Researcher}
  %~ \department{Faculty of Information Technologies}              %% \department is recommended
  \institution{Northeastern University}            %% \institution is required
  %~ \streetaddress{Street1 Address1}
  \city{Boston}
  %~ \state{State1}
  %~ \postcode{Post-Code1}
  \country{USA}                    %% \country is recommended
}
\email{pelenitsyn.a@northeastern.edu}          %% \email is recommended

\begin{abstract}
Julia is a dynamic language for scientific computing. For a dynamic language,
Julia is surprisingly typefull. Types are used not only to structure data but
also to guide dynamic dispatch – the main design tool in the language. No
matter the dynamism, Julia is performant: flexibility is compiled away at the
run time using a simple but smart type-specialization based optimization
technique called type stability. Based on a model of a JIT mimicking Julia
from previous works, we present the first algorithm to approximate type
stability of Julia code. Implementation and evaluation of the algorithm is
still a work in progress.
\end{abstract}
