\author{Artem Pelenitsyn}
\orcid{0000-0001-8334-8106}
\email{pelenitsyn.a@northeastern.edu}
\affiliation{%
  \institution{Northeastern University}
  \city{}
  \country{USA}
}

\begin{abstract}
Julia is a dynamic language for scientific computing. For a dynamic language,
Julia is surprisingly typeful. Types are used not only to structure data but
also to guide dynamic dispatch – the main design tool in the language. No
matter the dynamism, Julia is performant: flexibility is compiled away at the
run time using a simple but smart type-specialization based optimization
technique called type stability. Based on a model of a JIT mimicking Julia
from previous works, we present the first algorithm to approximate type
stability of Julia code. Implementation and evaluation of the algorithm is
still a work in progress.
\end{abstract}

\begin{CCSXML}
<ccs2012>
<concept>
<concept_id>10011007.10011006.10011041.10011044</concept_id>
<concept_desc>Software and its engineering~Just-in-time compilers</concept_desc>
<concept_significance>500</concept_significance>
</concept>
</ccs2012>
\end{CCSXML}

\ccsdesc[500]{Software and its engineering~Just-in-time compilers}

\keywords{method dispatch, type inference, compilation,%
  dynamic languages, Julia language}
