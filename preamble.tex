% ********************************************************************
%
%        Various Convenience Packages
%
% ********************************************************************
\usepackage[T1]{fontenc}
\usepackage{tcolorbox}
\usepackage{amsmath,amsthm}

\usepackage{xspace,url,hyperref,doi,wrapfig,stmaryrd,graphicx,xparse,etoolbox}
\usepackage{caption,subcaption}
\usepackage[inline]{enumitem}
\usepackage{cancel}
%\usepackage{showframe}  %% for checking margins
\definecolor{Gray}{gray}{0.9}
\definecolor{vlightgray}{gray}{0.93}
\interfootnotelinepenalty=10000 % don't break footnotes into several pages


% ********************************************************************
%
% Fine-tune hyperreferences (hyperref should be called last)
%
% ********************************************************************
% \hypersetup{%
%   %draft, % hyperref's draft mode, for printing see below
%   colorlinks=true, linktocpage=true, pdfstartpage=3, pdfstartview=FitV,%
%   % uncomment the following line if you want to have black links (e.g., for printing)
%   %colorlinks=false, linktocpage=false, pdfstartpage=3, pdfstartview=FitV, pdfborder={0 0 0},%
%   breaklinks=true, pageanchor=true,%
%   pdfpagemode=UseNone, %
%   % pdfpagemode=UseOutlines,%
%   plainpages=false, bookmarksnumbered, bookmarksopen=true, bookmarksopenlevel=1,%
%   hypertexnames=true, pdfhighlight=/O,%nesting=true,%frenchlinks,%
%   urlcolor=CTurl, linkcolor=CTlink, citecolor=CTcitation, %pagecolor=RoyalBlue,%
%   %urlcolor=Black, linkcolor=Black, citecolor=Black, %pagecolor=Black,%
%   pdftitle={\myTitle},%
%   pdfauthor={\myName},%
%   pdfsubject={\myTitle},%
%   pdfkeywords={\myKws},%
%   pdfcreator={pdfLaTeX},%
%   pdfproducer={LaTeX with hyperref and classicthesis}%
% }

% *********************************************************************************************
%
%               Setup code listings
%
% *********************************************************************************************

\usepackage{style/julia}
\usepackage{mathpartir,listings}
\lstdefinelanguage{Jules}{
  keywords={struct,is,end},
  keywordstyle=\color{darkgray}\bfseries,
  ndkeywords={struct,is,end},
  ndkeywordstyle=\color{darkgray}\bfseries,
  identifierstyle=\color{black},
  sensitive=false,  comment=[l]{//},  morecomment=[s]{/*}{*/},
  commentstyle=\color{gray}\ttfamily,  stringstyle=\color{gray}\ttfamily,
  morestring=[b]',  morestring=[b]",
  aboveskip=\medskipamount, %0em,
  belowskip=\medskipamount, %0em
  escapeinside={(*@}{@*)}
}
\lstset{language=julia}
\newcommand{\code}[1]{{\ttfamily #1}\xspace}
\renewcommand{\c}[1]{\lstinline[language=Julia]!#1!\xspace}
